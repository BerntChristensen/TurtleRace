% !TEX encoding = UTF-8 Unicode
\documentclass[a4paper]{article}


\usepackage[T1]{fontenc}     % För svenska bokstäver
\usepackage[utf8]{inputenc}  % Teckenkodning UTF8
\usepackage[swedish]{babel}  % För svensk avstavning och svenska
                             % rubriker (t ex "Innehållsförteckning")
\usepackage{fancyvrb}        % För programlistor med tabulatorer
\fvset{tabsize=4}            % Tabulatorpositioner
\fvset{fontsize=\small}      % Lagom storlek för programlistor

\title{TurtleRace \\
	Inlämningsuppgift 1, Programmeringsteknik för C/D}
\author{Bernt Christensen, C11 (dic11bch@student.lu.se)\\
Johan Bäversjö, C11 (dic11jba@student.lu.se)}


% *** Tillägg för denna rapport. ***
% Paket:
\usepackage{graphicx}         % För att inkludera bilder.

% Kommandon i denna rapport
\newcommand{\code}[1]{\texttt{#1}} % För programkod i text.
% *** Slut på tillägg för denna rapport. ***


\begin{document}              % Början på dokumentet

\maketitle                    % Skriver ut rubriken som vi
                              % definierade ovan med \title, \author
                              % och eventuellt \date
\section{Bakgrund}
Uppgiften består i att skapa en tävlingsbana som är oberoende av vad för tävling som utförs på den, en tävling där två stycken sköldpaddor tävlar mot varandra och att genomföra själva loppet där dessa två sköldpaddor möts. 

\section{Modell}
Modellen av loppet innehåller följande klasser:

\begin{tabular}{lp{8cm}}
\code{TurtleRace} & Innehåller \code{main}-metoden som initierar alla objekt som behövs för att genomföra ett lopp. \\
\code{Turtle} & Klass som beskriver en Turtle i planet. \\
\code{RaceTrack} & Klass som beskriver en banas utseende, är oberoende av vilken typ av objekt som tävlar. \\
\code{RacingEvent} & Beskriver en tävling. Håller reda på bana samt deltagare \\

\end{tabular}

\vspace{\baselineskip}
Simuleringen implementeras genom att programmet under hela simuleringstiden varje minut undersöker vad som ska ske. Följande händelser kan inträffa:

\begin{itemize}
\item En bil anländer till systemet. Detta hanteras av klassen \code{Entry}, som varje minut drar ett Poisson-fördelat slumptal som anger antalet bilar som anländer.
\item En bil börjar tvättas (flyttas från kön till en tvättstation). Detta hanteras av klassen \code{CarWash}.
\item En bil är färdigtvättad (flyttas från en tvättstation till utfarten). Också detta hanteras av klassen \code{CarWash}.
\item En bil lämnar systemet. Detta hanteras av klassen \code{Exit}.
\end{itemize}


De nämnda ''aktiva'' klasserna har alla en metod \code{tick()} som anropas varje minut. Även klassen \code{Clock}, som håller reda på tiden i systemet, har en sådan metod.

Bilarna (\code{Car}-objekten) är helt passiva och skickas bara runt i systemet av de aktiva klasserna. Klasserna \code{Queue} och \code{WashingStation} är också passiva. Deras uppgift är att hålla reda på \code{Car}-objekt under tiden bilarna står i kö eller tvättas. 


\section{Brister och kommentarer}
Programmet kan endast hantera ett lopp mellan två stycken sköldpaddor i dess nuvarande utförande. För att kunna lägga till sköldpaddor behövs stora redigeringar i koden göras.

En förbättring hade varit att använda en vektor istället för dynamiskt kunna bestämma antalet skäldpaddor.
Båda skölldpaddorna har okända namn, vilket medför att det inte är möjligt att ta reda på vilken sköldpadda som vunnit bortsett från det visuella resultatet.


\section{Programlistor}
Klasserna finns i filer med motsvarande namn. Till exempel innehåller filen  \code{RaceTrack.java} klassen \code{RaceTrack}. Alla klasser som används finns i samma katalog som huvudprogrammet

\subsection{\code{TurtleRace}}
\VerbatimInput{../src/TurtleRace.java}


\subsection{\code{Turtle}}
\VerbatimInput{../src/Turtle.java}

\subsection{\code{RaceTrack}}
\VerbatimInput{../src/RaceTrack.java}

\subsection{\code{RacingEvent}}
\VerbatimInput{../src/RacingEvent.java}


%\VerbatimInput{CarWash.java}

%\ldots\ osv, liknande programlistor för samtliga klasser.

\end{document}                  % Slut på dokumentet
